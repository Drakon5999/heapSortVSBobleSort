\documentclass[a4paper,12pt,titlepage,finall]{article}

\usepackage[T2A]{fontenc}     % форматы шрифтов
\usepackage[T1]{fontenc}
\usepackage[utf8x]{inputenc}     % кодировка символов, используемая в данном файле
\usepackage[russian]{babel}      % пакет русификации
\usepackage{tikz}                % для создания иллюстраций
\usepackage{pgfplots}            % для вывода графиков функций
\usepackage{geometry}		 % для настройки размера полей
\usepackage{indentfirst}         % для отступа в первом абзаце секции
\usepackage{multirow}            % для таблицы с результатами
\usepackage{mathtext}

% выбираем размер листа А4, все поля ставим по 2см
\geometry{a4paper,left=20mm,top=20mm,bottom=20mm,right=20mm}

\setcounter{secnumdepth}{0}      % отключаем нумерацию секций

\usepgfplotslibrary{fillbetween} % для изображения областей на графиках
%wtf
\begin{document}
% Титульный лист
\begin{titlepage}
    \begin{center}
	{\small \sc Московский государственный университет \\имени М.~В.~Ломоносова\\
	Факультет вычислительной математики и кибернетики\\}
	\vfill
	{\Large \sc Отчет по заданию №1}\\
	~\\
	{\large \bf <<Методы сортировки>>}\\ 
	~\\
	{\large \bf Вариант 2 / 4 / 1 / 5}
    \end{center}
    \begin{flushright}
	\vfill {Выполнил:\\
	студент 104 группы\\
	Лозинский~И.~П.\\
	~\\
	Преподаватель:\\
	Гуляев~Д.~А.}
    \end{flushright}
    \begin{center}
	\vfill
	{\small Москва\\2017}
    \end{center}
\end{titlepage}

% Автоматически генерируем оглавление на отдельной странице
\tableofcontents
\newpage

\section{Постановка задачи}

\begin{itemize}
\item Реализовать метод «пузырька» сортировки массива чисел
\item Реализовать пирамидальную сортировку массива чисел
\end{itemize}
Провести эксперементальное сравнение эффективности методов при условии, что:
\begin{itemize}
\item Массив содержит элементы типа «long long int»
\item Числа сортируются по невозрастанию их модуля
\end{itemize}
Анализ эффективности  производится путём подсчёта операций сравнения элементов и операций перестановки элементов.\\
Для сравнения использовать динамические массивы длиной 10, 100, 1000 и 10 000 элементов.\\
Для каждого колличества элементов сгенерировать:
\begin{itemize}
\item Массив отсортированный по возрастанию
\item Массив отсортированный по убыванию
\item 2 массива отсортированных случайным образом
\end{itemize}

\newpage

\section{Результаты экспериментов}
\subsection{Сортировка методом «пузырька»: ~\cite{cs}}
Теоретические оценки числа сравнений:
\begin{itemize}
\item Наилучший случай: ${N-1}$
\item Наихудший случай: $(N-1)\frac{N}{2}$
\end{itemize}
Теоретические оценки числа обменов:
\begin{itemize}
\item Наилучший случай: $0$
\item Наихудший случай: $(N-1)\frac{N}{2}$
\end{itemize}
Практические результаты представлены в таблице:
\begin{table}[h]
\centering
\begin{tabular}{|c|c|c|c|c|c|c|c|}
    \hline
    \multirow{2}{*}{\textbf{n}} & \multirow{2}{*}{\textbf{Параметр}} & \multicolumn{4}{|c|}{\textbf{Номер сгенерированного массива}} & \textbf{Среднее} \\
    \cline{3-6}
    & & \parbox{1.5cm}{\centering 1} & \parbox{1.5cm}{\centering 2} & \parbox{1.5cm}{\centering 3} & \parbox{1.5cm}{\centering 4} & \textbf{значение} \\
    \hline
    \multirow{2}{*}{10} & Сравнения & 90&9&72&81&63  \\
    \cline{2-7}
                        & Перемещения & 45&0&23&29&24 \\
    \hline
    \multirow{2}{*}{100} & Сравнения &9900&99&8514&8316&6707 \\
    \cline{2-7}
                         & Перемещения & 4950&0&2541&2674&2541 \\
    \hline
    \multirow{2}{*}{1000} & Сравнения &999000&999&949050&964035&728271 \\
    \cline{2-7}
                          & Перемещения &499500&0&248447&243591&247884 \\
    \hline
    \multirow{2}{*}{10000} & Сравнения &99990000&9999&99110088&99760023&74717527 \\
    \cline{2-7}
                           & Перемещения &49995000&0&25123967&25068395&25046840 \\
    \hline
\end{tabular}
\caption{Результаты работы сортировки методом «пузырька»}
\end{table}

\newpage

\subsection{Пирамидальная сортировка: ~\cite{css}}
Теоретические оценки числа сравнений:
\begin{itemize}
\item Наилучший случай: ${N\log_{}{N}}$
\item Наихудший случай: ${N\log_{}{N}}$
\end{itemize}
Теоретические оценки числа обменов:
\begin{itemize}
\item Наилучший случай: ${\frac{N}{2}\log_{}{N}}$
\item Наихудший случай:  ${N\log_{}{N}}$
\end{itemize}
Практические результаты представлены в таблице:
\begin{table}[h]
\centering
\begin{tabular}{|c|c|c|c|c|c|c|c|}
    \hline
    \multirow{2}{*}{\textbf{n}} & \multirow{2}{*}{\textbf{Параметр}} & \multicolumn{4}{|c|}{\textbf{Номер сгенерированного массива}} & \textbf{Среднее} \\
    \cline{3-6}
    & & \parbox{1.5cm}{\centering 1} & \parbox{1.5cm}{\centering 2} & \parbox{1.5cm}{\centering 3} & \parbox{1.5cm}{\centering 4} & \textbf{значение} \\
    \hline
    \multirow{2}{*}{10} & Сравнения &35&41&34&37&36  \\
    \cline{2-7}
                        & Перемещения &21&30&25&28&26 \\
    \hline
    \multirow{2}{*}{100} & Сравнения &944&1081&1020&1037&1020 \\
    \cline{2-7}
                         & Перемещения &516&640&572&590&579\\
    \hline
    \multirow{2}{*}{1000} & Сравнения &15965&17583&16897&16794&16809 \\
    \cline{2-7}
                          & Перемещения &8316&9708&9097&9045&9041\\
    \hline
    \multirow{2}{*}{10000} & Сравнения &226682&244460&235373&235358&235468\\
    \cline{2-7}
                           & Перемещения &116696&131956&124162&124191&124251\\
    \hline
\end{tabular}
\caption{Результаты работы пирамидальной сортировки}
\end{table}
\newpage

\section{Структура программы и спецификация функций}
\begin{itemize}
\item \ttfamily{void \underline{genArray}(size\_t n, ll *arr, int order)}\rmfamily – генерирует массив из \ttfamily{n} \rmfamily элементов в порядке
\ttfamily{order} \rmfamily и сохраняет его по указателю \ttfamily{arr}\rmfamily.\\
\ttfamily{order} \rmfamily принимает следующие значения:
    \begin{itemize}
    \item -1 – массив отсортирован по убыванию
    \item 1 – массив отсортирован по возрастанию
    \item 0 – массив отсортирован по случайно
    \end{itemize}

\item \ttfamily{void \underline{swap}(ll *a, ll *b)}\rmfamily – меняет местами значения 2х указателей \ttfamily{a} \rmfamily и \ttfamily{b} \rmfamily типа \textit{long long int}
\rmfamily и увеличивает значение счётчика \ttfamily{swaps}

\item \ttfamily{bool \underline{isLess}(ll a, ll b)} \rmfamily – сравнивает модули аргументов и увеличивает счётчик \ttfamily{compares}\rmfamily. Возвращает \ttfamily{True} \rmfamily если \ttfamily{|a| < |b|}\rmfamily.
\item \ttfamily{void \underline{reset}(void)} \rmfamily – сбрасывает значения счётчиков \ttfamily{compares} \rmfamily и \ttfamily{swaps} \rmfamily.
\item \ttfamily{void \underline{bubbleSort}(size\_t n, ll *arr)} \rmfamily – реализует сортировку методом «пузырька»
\item \ttfamily{void \underline{heapify}(size\_t i, size\_t n, ll *arr)} \rmfamily – восстанавливает свойство кучи для элемента с идексом \ttfamily{i} \rmfamily массива \ttfamily{arr} \rmfamily из \ttfamily{n} \rmfamily элементов.
\item \ttfamily{void \underline{heapSort}(size\_t n, ll *arr)} \rmfamily – реализует пирамидальную сортировку
\item \ttfamily{void \underline{printarr}(size\_t n, ll *arr)} \rmfamily – отладочный метод для распечатки массива
\item \ttfamily{void \underline{compareSorts}(int type, size\_t n, unsigned int *resultbc, unsigned int *resulthc,\\
                  unsigned int *resultbs, unsigned int *resulths)} \rmfamily – проводит тестирование сортировок на массивах типа \ttfamily{j} \rmfamily (см. параметр  \ttfamily{order} \rmfamily в функции \ttfamily{genArray}\rmfamily) и длинны \ttfamily{n}\rmfamily.\\ Результаты записывает по переданным указателям:
     \begin{itemize}
     \item \ttfamily{resultbc} \rmfamily – количество сравнений в методе «пузырька»
     \item \ttfamily{resulthc} \rmfamily – количество сравнений в пирамидальной сортировке
     \item \ttfamily{resultbs} \rmfamily – количество перестановок элементов в методе «пузырька»
     \item \ttfamily{resulths} \rmfamily – количество перестановок элементов в пирамидальной сортировке
     \end{itemize}
\item \ttfamily{bool \underline{test}()} \rmfamily – запускает 3 раза методы сортировки на случайных массивах длинны 1000, если в результате какой-либо из массивов оказался не отсортирован – возвращает \ttfamily{False} \rmfamily, иначе \ttfamily{True}\rmfamily.
\item \ttfamily{int \underline{main}(int argc, char *argv[])} \rmfamily – при запуске без аргументов вывдит читаемый текст с результатами, иначе текст для вставки в TeX отчёт.
\end{itemize}
\newpage

\section{Отладка программы, тестирование функций}

Тестирование функций сортировки производится при помощи  \ttfamily{test()}\rmfamily.\\
Отладочный вывод массивов производится при помощи функции  \ttfamily{printarr()}\rmfamily.

\newpage

\section{Анализ допущенных ошибок}
Полученные практические результаты подтверждают теоретические оценки.\\
Однако для более корректной оценки среднего числа сравнений/перемещений элементов следует провести больше тестов на случайных входных данных.\\
Возможна погрешность связанная с выбранным способом генерации случайных массивов, т.к. он далеко не идеален.


\newpage
\begin{raggedright}
\addcontentsline{toc}{section}{Список цитируемой литературы}
\begin{thebibliography}{99}
\bibitem{css} Кормен Т., Лейзерсон Ч., Ривест Р, Штайн К. Алгоритмы: построение и анализ.
    Второе издание.~--- М.:<<Вильямс>>, 2005.
\bibitem{cs} Лорин Г. Сортировка и системы сортировки. — М.: Наука, 1983.
\end{thebibliography}
\end{raggedright}


\end{document}
